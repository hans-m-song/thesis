\chapter{Introduction}

% \todo[inline]{
%     general aims – what you intend to contribute to the understanding of a topic \\
%         - Decentralization of computation in mining \\
%         - Decentralization of control
%     specific objectives – which particular aspects of that topic you'll be investigating \\
%         - Decentralization of stake \\
%         - Simulation/model as proof
%     the rationale for proceeding in the way that you did \\
%         - ? \\
%     your motivation or the justification for your research – the level of detail can vary depending on how much detail you will be including in a literature review. \\
%         - Current situation with BitCoin \\
%         - Previous case with ghash.io
% }

At it's core, blockchain is about creating a public ledger that ensures immutability through decentralisation. While there have been many implementations, all platforms require overcoming the issue of reaching consensus in a distributed trust-less environment. Bitcoin \cite{nakamoto2009} is the platform which spurred the demand for innovation in blockchains with Satoshi Nakamoto's innovative Proof of Work (PoW) consensus algorithm. Bitcoin now stands at the forefront of cryptocurrencies with market capitalisation fluctuating around 100 billion USD \cite{bitcoinmarketcap2020}, utilising an estimated 55TWh annually \cite{cambridge2020} as of November 2020.

The security of the PoW consensus algorithm (refer to \cref{section:pow}) is established through computationally expensive cryptographic puzzles to prove the ``work" contributed to the validation of a block. These are solved by \textit{miners}, people or companies that dedicate computational power to guess the solution for a reward. Dedicating more resources gives miners a higher probability of guessing the solution. As blockchains utilising PoW rose in popularity, the number of competing miners seeking to profit from the rewards rose alongside. While there are still individual miners, the bulk of computation stems from miners that have contributed their computational power to a \textit{pool}, a congregation of miners lead by a pool supervisor who coordinates work and distributes rewards. Amortised, this would result in similar payouts to mining individually, with smaller payout and increased frequency, resulting in reduced variance. However, this also leads to large portions of computational power aggregated under the authority of singular entities, effectively centralising computation.

With four pools contributing almost 60\% of the total hash rate for Bitcoin \cite{bitcoinpools2020} at the time of this writing and a real possibility for pools to exceeded 50\% (such as GHash \cite{ghash2019} in 2014). The only limiting factor is the pools goodwill, potential public backlash and a shared incentive to maintain the reputation and valuation of the currency. As each pool has a centralised authority orchestrating operations, the scenario in which they collude to execute a 51\% attack is a possibility. Additionally, pool supervisors are able to censor transactions by excluding them from the blocks they mine. This trend and it's potential effects has been investigated previously to through varying methods \cite{oceanic2020} \cite{centralisation2015}.

This project introduces and verifies an economic model inspired by the Harberger Tax (refer to \cref{section:harberger-tax}) that has a focus on encouraging the separation the authority from pool supervisors to external participants, negating the vulnerability introduced by the centralisation of authority. This is achieved by creating a sharing economy (refer to \cref{section:strategic-game}) in which individual miners or pools can offer their computational power to a public marketplace as \textit{chunks} from which external participants or other miners may purchase to become \textit{owners}. Through this marketplace, miners will be consistently rewarded with taxation on their contributed power while owners can have a chance at the block reward without the hardware or power required. Additionally, this means that pools no longer need to adhere to the rule of having less than 50\% of the global hash rate as the authority is in the hands of the owners. The aim of the model is to attract pools to utilise the model as contributors (\textit{miners}) rather than participants (\textit{owners}). In order to verify the economic model, it will first be formally defined and modelled with a deterministic state generator. Then, a simulation implementing the model will be created in order to optimise parameters and to observe any potential predisposed trends. Using the optimal parameters, the model's performance can be assessed through the actions of the participants as they make decisions purchasing chunks within the simulation.
