Proof of Work blockchains are write-only ledgers existing on peer to peer networks, requiring new data to go through a computationally expensive validation process by \textit{miners} who dedicate their hardware in return for financial compensation. While a core principle of blockchains is to be decentralised, the vast majority of miners choose to form coalitions and pool their computational power, negating the decentralised nature of the blockchain. 

In this paper, the concept of computation was gamified through an economic model with a focus on incentivising the distribution computational power through the application of Harberger Tax and auction mechanisms is proposed and formalised. A state generator was created based on a simplified deterministic version of the model to formalise potential strategies, observe expected behaviour, and make predictions of expected rewards based on participant strategies. A simulation was then created to validate the predictions against a more realistic implementation of the model. These tools revealed that while there are some flaws in the model, it has potential to incentivise the distribution of computational power. Finally, further areas of investigation have been identified for future research.
