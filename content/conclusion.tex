\chapter{Conclusion and Future Work}

% \todo[inline]{
%     emphasise that you've met your research aims \\
%     summarise the main findings of your research \\
%     restate the limitations of your research and make suggestions for further research.
% }

The Harberger Tax provides a viable economy as a starting point in ideal conditions. The probability of achieving the equilibrium state and maximising expected reward is heavily dependent on the willingness of a participant to actively engage in trading. However there exist limitations in terms of scalability and longevity as maintaining fairness requires balancing parameters such as taxation, trade interval versus block time, and the number of chunks per participant. Additionally, The stability and distributed nature is heavily reliant on implementation-specific mechanisms and the effects of integrating with a system that allocates computational power is still unknown.

While the results and analysis could have been more in-depth, there were some roadblocks encountered due to both hardware and experimental design limitations. Hence, there are several avenues of further investigation:

\begin{itemize}
    \item Extend state diagram generator to reach final states given hardware limitations (perhaps through a database or different language).
    \item Increasing complexity of simulation by allowing participants to join mid-game, allowing multi-chunk purchases, or by implementing different auction formats.
    \item Improve modelling of participants decision-making process.
    \item Explore mechanisms to facilitate a real-world implementation with a decentralised orchestrator (as a smart contract) or as a centralised authority.
    \item Using real-world values to evaluate the true profit or loss any participating entity may experience.
\end{itemize}

